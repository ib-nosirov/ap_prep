\documentclass[11pt]{article}
\usepackage{xcolor}

\title{AP Literature Q2 Template 2020: An Argument in Three Acts}

\begin{document}
\maketitle

\section{A Strategy for Tackling the Q2}
Throughout the year, I've read and written quite a few timed writes. 
My scores have lacked consistency and I've struggled to figure out why.
Having read some recent student response samples, it dawned me that the AP exam doesn't grade novelty in your \textit{interpretation}, but rather your \textit{understanding} of the text.
To explain what I mean, here is a passage from \textit{Hamlet}:
\begin{quote}
	\small{O, that this too, too sullied flesh would melt,\\
Thaw, and resolve itself into a dew,\\
Or that the Everlasting had not fixed\\
His canon ’gainst self-slaughter! O God, God,\\
How weary, stale, flat, and unprofitable\\
Seem to me all the uses of this world!\\
Fie on ’t, ah fie! ’Tis an unweeded garden\\
That grows to seed. Things rank and gross in nature\\
Possess it merely. That it should come to this:\\
But two months dead—nay, not so much, not two.\\
So excellent a king, that was to this\\
Hyperion to a satyr; so loving to my mother\\
That he might not beteem the winds of heaven\\
Visit her face too roughly. Heaven and Earth,\\
Must I remember? Why, she would hang on him\\
As if increase of appetite had grown\\
By what it fed on. And yet, within a month\\
(Let me not think on ’t; frailty, thy name is woman!),\\
A little month, or ere those shoes were old\\
With which she followed my poor father’s body,\\
Like Niobe, all tears—why she, even she\\
(O God, a beast that wants discourse of reason\\
Would have mourned longer!), married with my\\
uncle,\\
My father’s brother, but no more like my father\\
Than I to Hercules. Within a month,\\
Ere yet the salt of most unrighteous tears\\
Had left the flushing in her gallèd eyes,\\
She married. O, most wicked speed, to post\\
With such dexterity to incestuous sheets!\\
It is not, nor it cannot come to good.\\
	But break, my heart, for I must hold my tongue.\\}
\end{quote}

A prompt might ask how Shakespeare uses literary devices to convey Hamlet's opinions.
A few months ago, I would have treated this as a scavenger hunt for different literary devices.
I would have probably had a complex \textit{sounding} thesis statement spanning 3 lines and tried to leverage that to reach for the complexity point, which I ultimately wouldn't get.
While my sentence structure and word choice would probably be fine, with the exception of colloquial language and adjectives that improperly gauge the intended connotation, the essay as a whole would not flow very well and I would struggle with transitions.
The reason is that up until very recently, I've organized my essays based on what Shakespeare is saying rather than what Hamlet is saying.
I would try to read between the lines to \textit{interpret} text as a whole --- something that I've found is not feasible, hence not expected, in a 40 minute essay.
It also comes as no surprise that my timed writes would sometimes sound too much like blog posts.
To score well on a passage prompt, you have to start with carefully analyzing how the speaker is feeling, what they are thinking, and how those two things shift throughout the passage.
This sounds an awful lot like telling a story ---
a story at the end of which you get to stand up in front of your audience (the AP Grader) and tell them the moral of the story based \textit{solely} on what you wrote about in the previous 4 paragraphs.
With that said, here is a template that I think will help us do well on the exam next week.\\
\section{Introduction}
This section is a good place to pinpoint exactly how the character(s) is/are feeling and what they is/are thinking while quickly summarizing what's going on in the scene to prove that you are indeed literate.
You can tell from Hamlet's suicidal intro that he is extremely upset with his mother's actions. Period. Leave it at that.
In the next sentence talk about a shift (based on how much Carlos talked about the shift, I think there's bound to be one next week).
Hamlet's thoughts shift from praising his father's virtue to condemning his mother's lack of virtue.
For the \textit{Hamlet} passage I will use the thesis, \color{blue} ``Shakespeare juxtaposes Hamlet's perceptions of his mother and his father to convey Hamlet's disbelief at his mother's quick marriage, which he perceives to be a violation of universal virtue."
\color{black} As I write this, I'm super tempted to write that his mother, a family member, violating a code that he deems universal to all humans makes him question his faith in humanity.
This is an interpretation and even if I somehow had the time to argue this point, I wouldn't be saying anything new throughout the entire essay.
Instead, I'm going to save the `family member' card for now, touch on it briefly in the second to last paragraph, and really bring it home for that complexity point in the end.
I already have things to argue: `human virtue' and `disbelief.' 

\section{Claim 1/Act 1}
Using the first part of the \textit{Hamlet} passage, you could say something like, \color{blue}``Hamlet feels extremely distressed at his father's death because he feels that it is a sign that there is no justice in the world."
\color{black} Before I can argue that his mom violated what he deems is universal human virtue, I need to set up that Hamlet even believes in such a concept.
I start with how Hamlet is feeling \textit{at the start of the passage} and force myself to use the word ``because" to make sure I make a claim and not an observation. 
That way, I can use 2-3 sentences to talk about what in the passage references suicide and use another 2-3 sentences of evidence and commentary to talk about `no justice in the world.'
\color{blue} To finish off a successful paragraph, I tie back what he's feeling (distress) to the thesis and use that connection to transition into something else I put in the thesis: his perception of his father.\\
\color{black}
\section{Claim 2/Act 2}
I have to center this piece around perception because that's what I wrote in my thesis, so I'm going to say, \color{blue} ``Hamlet's tone shifts from disappointed lament to that of appraisal, signifying that he holds his father in unusually high regard."
\color{black}This doesn't seem like claim, and it wouldn't be if normal people compared their fathers to the titan of light that controls the heavenly winds.
This paragraph is all about the word ''unusually" and how Hamlet's regard for his father is so unusual, that it is enough to make him suicidal.
I argue about the tone using how Hamlet bemoans the Everlasting being against suicide and then using the winds of heaven as a double positive to gas up his dad's gentleman manners.
\color{blue} I use an embedded quote in the last sentence to draw attention the word satyr and how if Hamlet uses Hyperion to talk about his dad, he's using satyr to show how he feels about his mom. Another smooth transition, and tying back to the juxtaposition in the thesis.
\color{black}
\section{Claim 3/Act 3}
Typically in a play, Act 3 is where everything goes down. I've just argued human virtue, but still have to hammer home the fact that Hamlet feels disbelief.\color{blue} ``Hamlet references similar allusions he used to praise his father to condemn his mother and uncle, showing that he is in awe of the disparity in character between family members."
\color{black} Now I spend the rest of the time talking about what allusions he uses for Claudius (opposite of Hercules and Hyperion) and Gertrude (a hypocrite and a beast, who is worse than a mythological being --- very unlike Hyperion). In the last few paragraphs, I emphasize that they are all part of one family, which makes it all the more difficult for Hamlet.

\section{Conclusion/Show-and-Tell}
This is where I talk about how the fact that Hamlet believes so earnestly in justice, the fact that he holds his father in unusually high esteem, and the fact that all three people are of the same family rocks Hamlet to the core, challenging all of his beliefs about goodness in the world. I end by saying that all of this disbelief makes him wish that he could escape such an evil world, a plea that he echoes at the start of the passage, showing that he is growing unstable as he grapples with this situation.\\

I spent 2 hours writing this, so even if I can manage to get the general flow down in 40 minutes, the essay is bound to get a decent score. Hope you found this guide useful.

\end{document}
