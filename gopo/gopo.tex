\documentclass[12pt]{article}
\usepackage{hyperref}
\hypersetup{
    colorlinks=true,
    linkcolor=blue,
    filecolor=magenta,      
    urlcolor=cyan,
}
\usepackage[margin=0.5in]{geometry}

\title{AP United States Government and Politics Study Guide 2020}
\date{ }
\begin{document}
\maketitle

\section*{Theories of Representative Democracy}
	\textbf{Pluralist} - competing groups are healthy because they provide a connectiong to government; competition clarifies information and prevents monopoly.\\
	\textbf{Elite} - power is concentrated in the largest and riches organizations; ultimately money talks, and these large groups will have the most influence.\\
	\textbf{Participatory} - democracy that relies on majority rule and is against representative democracy at all; can lead to oppression of the minority.\\
	US Government has a combo of all 3\\

\section*{Types of Federalism}
\textbf{Dual Federalism (Layer Cake)} - Prevalent from the Civil War period to New Deal (1930s); federalism characterized by a national government exercising its powrs independently from state governments. \\
\textbf{Cooperative Federalism (Marble Cake)} - The federal government becomes more intrusive in what had been typically the domain of state governments. The national government provided the money, and state governments administered the programs.Some programs of this form of federalism, like the AAA, were deemed unconstitutional.\\
\textbf{Fiscal Federalism (Creative/Competitive)} - States compete for pieces of governmental support and accept them under conditions and with a promise to develop programs of their own. This leads to Grants: categorical grants, block grants, and revenue sharing. The overall objective is to provide states with funds they normally wouldn't get to create programs that will be self-sufficient later on so the federal government doesn't end up with a massive bureaucracy.\\

%Constitution%
\section*{Constitution}

\subsection*{Article 1}
\begin{itemize}
	\item A House member must be at least 25 year old, an American citizen for seven years, and an inhabitant of the state the representative represents. Representatives serve two-year terms.
	\item A Senate member must be 30 years old, an American citizen for nine years, and a resident of the state the senator represents. Senators serve six-year terms.\\

		\textit{Common powers delegated to Congress, listed in Article I section 8:}
	\item power to tax
	\item coin money
	\item declare war
	\item Elastic Clause: "Congress shall the right to make all laws which shall be necessary and proper for carrying into execution the foregoing powers, and all other powers vested by this constitution." \hyperlink{McCulloch}{McCulloch v. Maryland (1819)} 
	\item Commerce Clause: Gives Congress the right ``to regulate Commerce with foreign Nations, and among the several States, and with Indian Tribes."\hyperlink{Lopez}{United States v. Lopez (1995)}
	\item The House of Representatives has the power to begin all revenue bills, to select a president if there is no Electoral College majority, and to initiate impeachment proceedings.
	\item Senate has the power to approve presidential appointments and treaites and to try impeachment proceedings.
	\item Congress may overrule a presidential veto by a two-thirds vote of each house.
\end{itemize}

\subsection*{Article 2}
	\begin{itemize}
		\item Must be 35 years old, a natural-born citizen, and a resident of the United States for 14 years.
		\item Chief executive
		\item Commander in Chief of the armed forces
		\item Power to grant pardons
		\item Power to make treaties
		\item Power to appoint ambassadors, justices, and other officials
		\item Power to sign legislation or veto legislation
		\item Duty to give a State of the Union report
		\item Election by Electoral College
		\item Definition of term limits, order of succession, and procedures to follow during presidential disability through constitutional amendments
		\item Informal power based on precedent, custom, and tradition in issuing executive orders (orders initiated by the president that do not require congressional approval), executive privilege (keeping executive meetings private), signing statements (presidential statements made in conjunctions with a president signing a bill), and creating executive agencies
	\end{itemize}

\subsection*{Article 3}
	\begin{itemize}
		\item Judges are appointed by the president with the consent of the Senate and serve for life, based on good behavior.
		\item Judicial power extends to issues dealing with common law, equity, civil law, criminal law, and public law.
		\item Cases are decided through original jurisdiction or appellate jurisdiction.
		\item The Chief Justice of the Supreme Court presides over impeachment trials.
		\item Congress creates courts that the `` inferior to the Supreme Court.
	\end{itemize}
%Bill of Rights%
\section*{Bill of Rights}
\subsection*{1st}
	\begin{itemize}
		\item \textbf{Freedom of Religion/Establishment Clause}: Congress shall make no law respecting an establishment of Religion. \hyperlink{Engel}{Engel v. Vitale (1962)}
		\item \textbf{Free Exercise Clause}: ...or prohibiting the free exercise thereof. \hyperlink{Wisconsin}{Wisconsin v. Yoder (1972)}
		\item \textbf{Freedom of Speech and the Press}: ...or abridging the freedom of speech, or of the press. \hyperlink{Schenck}{Schenck v. United States (1919)}, \hyperlink{Tinker}{Tinker v. Des Moines (1969)}, \hyperlink{NYT Co.}{New York Times v. United States (1971)}
		\item \textbf{Freedom of Assembly}:or the right of the people peaceably to assemble, and to petition the Government for a redress of grievances.
	\end{itemize}

\subsection*{2nd}
	\begin{itemize}
		\item The right of the people to keep and bear arms, shall not be infringed. \hyperlink{McDonald}{McDonald v. Chicago (2010)}
	\end{itemize}
\subsection*{4th}
	\begin{itemize}
		\item prevents unreasonable search and seizures. Right to privacy. \hyperlink{Roe}{Roe v. Wade (1965)}
		\item 4th, 5th, 6th, and 7th amendments all deal with due process.
	\end{itemize}
\subsection*{5th}
	\begin{itemize}
		\item no person shall be held to answer for a capital crime unless a Grand Jury is present.
		\item no double jeapordy.
		\item no person shall be compelled to be a witness against himself, or be deprived of life, liberty, or property without due process of law.
		\item 4th, 5th, 6th, and 7th amendments all deal with due process.
	\end{itemize}
\subsection*{6th}
	\begin{itemize}
		\item speedy and public trial.
		\item informed of the nature of the accusations.
		\item right to witnesses.
		\item Assistance of Counsel for his defense. \hyperlink{Gideon}{Gideon v. Wainwright (1964)}
	\end{itemize}
\subsection*{7th}
	\begin{itemize}
		\item the right of trial by jury shall be preserved.
	\end{itemize}

\subsection*{8th}
	\begin{itemize}
		\item no cruel and unusual punishments.
	\end{itemize}

\subsection*{9th}
	\begin{itemize}
		\item Rights not defined anywhere in the Constitution are retained by the people.
	\end{itemize}

\subsection*{10th}
	\begin{itemize}
		\item powers not given to the Federal Government are reserved to states.
	\end{itemize}
\subsection*{14th}
	\begin{itemize}
		\item Guarantees citizens protection from abuses by the federal government.
		\item ``No State shall make or enforce any law which shall abridge the privileges or immunities of citizens of the United States;
			nor shall any State deprive any person of life, liberty, or property without due process (\hyperlink{Roe}{Roe v. Wade (1973)}), (\hyperlink{Gideon}{Gideon v. Wainright (1964)}), (\hyperlink{McDonald}{McDonald v. Chicago (2010)}) of law;
			nor deny to any person within its jurisdiction the equal protection of the laws."\hyperlink{Brown}{Brown v. Board of Education (1954)}, \hyperlink{Shaw}{Shaw v. Reno (1993)}
	\end{itemize}

\subsection*{15th}
	\begin{itemize}
		\item prohibits the federal government and the states from discriminating the right to vote based on "race, color, or previous condition of servitude."
		\item De facto segregation made this useless until the Voting Rights Act of 1965.
		\item Voting Rights Act of 1965 - addressed poll tax and literacy requirements; gave attorney general the power to determine which states were in violation of the law; prohibited states from passing their own preclearance laws.\hyperlink{Shaw}{Shaw v. Reno (1993)}
	\end{itemize}
%supreme court cases%

\section*{SCOTUS Cases}
\hypertarget{McCulloch}{\subsection*{McCulloch v. Maryland (1819)}}
	\begin{itemize}
	\item Ruled that the National Bank was constitutional as per the "Necessary and Proper Clause" (Article 1 Section 8).
	\item Also found that individual states did not have the power to tax the national government because of the Supremacy Clause(Article 6 Section 2).
	\end{itemize}

\hypertarget{Lopez}{\subsection*{United States v. Lopez (1995)}}
	\begin{itemize}
		\item The enforcement of the Gun-Free School Zone Safety Act comes under the authority of the states and outside of the Commerce Clause.
	\end{itemize}

\hypertarget{Baker}{\subsection*{Baker v. Carr (1962)}}
	\begin{itemize}
		\item Decided on the question of whether federal courts could decide the political issue of state apportionment procedures. Court ruled that they could and stated that districts should be drawn based on population and not along party lines.
	\end{itemize}

\hypertarget{Shaw}{\subsection*{Shaw v. Reno (1993)}}
	\begin{itemize}
		\item Ruled that the bizarre shape of a particular district was motivated by racial partitioning and violated the equal protections clause of the 14th Amendment and the Voting Rights Act of 1965.
	\end{itemize}

\hypertarget{Schenck}{\subsection*{Schenck v. United States (1919)}}
	\begin{itemize}
		\item Ruled that the government can censor speech and press against the United States when there is a ``clear and present danger."
		\item Ruled that the Espionage Act was constitutional.
	\end{itemize}

\hypertarget{Brown}{\subsection*{Brown v. Board of Education (1954)}}
	\begin{itemize}
		\item Ruled that race-based segregation violated the equal protection clause of the 14th Amendment and ordered schools to be integrated.
		\item The following year, the Supreme Court ordered all segregated schools to integrate ``with all deliberate speed."
	\end{itemize}

\hypertarget{Engel}{\subsection*{Engel v. Vitale (1962)}}
	\begin{itemize}
		\item Ruled that New York State's encouragement of the recitation of prayer violates the Establishment Clause of the First Amendment.
	\end{itemize}

\hypertarget{Gideon}{\subsection*{Gideon v. Wainwright (1963)}}
	\begin{itemize}
		\item Ruled that through the 14th and 6th amendments, the accused had a right to a state appointed attorney even if he or she cannot afford one.
	\end{itemize}

\hypertarget{Tinker}{\subsection*{Tinker v. Des Moines (1969)}}
	\begin{itemize}
		\item Ruled that because of the Free Speech clause and the Due Process Clause of the 14th Amendment, symbolic speech was protected.
		\item Students wearing armbands in protest of the Vietnam War would not have their rights shed at the schoolhouse gates.
	\end{itemize}

\hypertarget{NYT Co.}{\subsection*{New York Times Company v. United States (1971)}}
	\begin{itemize}
		\item Ruled that the Freedom of Speech clause protects \textit{New York Times} publishing secret information that the government claims would be damaging to national security.
		\item This case was different than Schenck v. United States because the government couldn't present legitimate evidence for why the information was a threat to national security.
		\item Case famously stemming from the Pentagon Papers.
	\end{itemize}

\hypertarget{Wisconsin}{\subsection*{Wisconsin v. Yoder (1972)}}
	\begin{itemize}
		\item Ruled that mandating Amish students to attend school past the 8th grade was a violation of the free exercise clause of the First Amendment.
	\end{itemize}

\hypertarget{Roe}{\subsection*{Roe v. Wade (1973)}}
	\begin{itemize}
		\item Ruled that abortions are constitutionally protected. It set up a trimester system allowing gradual restrictions across trimesters of pregnancy, culminating in a permission to ban abortion in the third trimester.
	\end{itemize}

\hypertarget{McDonald}{\subsection*{McDonald v. Chicago (2010)}}
	\begin{itemize}
		\item Chicago maintained its gun control laws even after a previous case (\textit{District of Columbia v. Heller}).
		\item Court ruled that the 14th Amendment makes the 2nd Amendment right to keep and bear arms fully applicable to states.
	\end{itemize}
\hypertarget{Marbury}{\subsection*{Marbury v. Madison}}
	\begin{itemize}
		\item Judicial Review; established SCOTUS as a body that could declare a law as unconstitutional.
	\end{itemize}
\section*{Required Foundational Documents}
\subsection*{Federalist No. 10}
	Madison thought that political factions were a toxin to democracy, but also that they were inevitable and a permanent feature of democracy. He believed that the separation of power into three branches that would provide enough protection from those interests. He also believed that political factions would counterbalance private economic interests.\\
``By a faction, I understand a number of citizens... who are united by some common interest, adversed to the rights of other citizens, or to the permanent and aggregate interests of the community." ``the regulation of these various and interfering interests forms the principle task of modern Legislation."\\

\subsection*{Brutus No. 1}
	The antifederalists thought that the central government posed in the Constitution, even with its checks and balances was too powerful and that it would become tyrannial and would take away people's rights. Insisted on a Bill of Rights. 
	``in a republic as vast as the United States, the legislature cannot attend to the various concerns and wants of its different parts." Argument for smaller government.\\
``Many instances can be produced in which the people voluntarily increased the power of their rulers; but few, if any, in which rulers willingly abridged their authority."\\

\subsection*{Declaration of Independence}
	\begin{itemize}
		\item Philosophical Basis - Uses Locke's philosophy; ``unalienable rights;" limited governments are formed receiving powers from the ``consent of the governed."
		\item The Grievances - Taxation without representation; unjust trials; quartering of British soldiers; abolition of colonial assemblies; policy of mercantilism.
		\item The Statement of Separation - not only a right but a duty to change the government.
	\end{itemize}
\subsection*{Articles of Confederation}
	\begin{itemize}
		\item unicameral and weak national government without the ability to tax or hold a national army.
		\item strong state governments
		\item Congress given abilities to declare war, make peace, and sign treaties, borrow money.
		\item no chief executive or judicial system, legislature needed to pass with 2/3 majority.
		\item states could impose tariffs on each other and create their own currency.
		\item an amendment needed unanimous state support.
	\end{itemize}

\subsection*{Federalist No. 51}
	Madison's big idea was of checks and balances and of competing policy-making interests; every branch but the judicial branch would be elected by the people (including Senate with the 17th Amendment).\\
``If men were angels, no government would be necessary."\\

\subsection*{Federalist No. 70}
	According to Hamilton, a single president is necessary to ensure accountability in government, to enable the president to defend against legislative encroachments on his power, and to ensure ``energy" in the executive.\\
``The ingredients which constitute energy in the Executive are, first, unity; secondly, duration; thirdly, an adequate provision for its support; fourthly, competent powers. The ingredients which constitute safety in the republican sense are, first a due dependence on the people, secondly a due responsibility."\\
\subsection*{Federalist No. 78}
	Written by Hamilton to justify the structure of the judiciary under the Constitution. It needed to be the farthest away from the people and the members needed qualifications to be appointed for life. Having a good salary and lifelong tenure would prevent them from the pressures of politics and they could act independently. It would serve as the arbiter between executive and legislative while also being the weakest branch of government.\\
``The complete independence of the courts of justice is peculiarly essential in a limited Constitution...If, then, the courts of justice are to be considered as the bulwarks of a limited Constitution against legislative encroachments, this consideration will afford a strong argument for the permanent tenure of judicial offices, since nothing will contribute so much as this to that independent spirit in the judges which must be essential to the faithful performance of so arduous a duty."\\
\subsection*{Letter from a Birmingham Jail}
	It is the moral responsibility of the citizens to abide by morally just laws and its also their responsibility to break laws that are not morally right; a justification of the non-violent Civil Rights movement against racism.\\


\end{document}
